%%%%%%%%%%%%%%%%%%%%%%%%%%%%%%%%%%%%%%%%%
% Short Sectioned Assignment
% LaTeX Template
% Version 1.0 (5/5/12)
%
% This template has been downloaded from:
% http://www.LaTeXTemplates.com
%
% Original author:
% Frits Wenneker (http://www.howtotex.com)
%
% License:
% CC BY-NC-SA 3.0 (http://creativecommons.org/licenses/by-nc-sa/3.0/)
%
%%%%%%%%%%%%%%%%%%%%%%%%%%%%%%%%%%%%%%%%%

%----------------------------------------------------------------------------------------
%	PACKAGES AND OTHER DOCUMENT CONFIGURATIONS
%----------------------------------------------------------------------------------------

\documentclass[paper=a4, fontsize=11pt]{scrartcl} % A4 paper and 11pt font size

\usepackage[T1]{fontenc} % Use 8-bit encoding that has 256 glyphs
\usepackage{fourier} % Use the Adobe Utopia font for the document - comment this line to return to the LaTeX default
%\usepackage[spanish]{babel} % English language/hyphenation
\usepackage[utf8]{inputenc}
\usepackage{amsmath,amsfonts,amsthm} % Math packages
\usepackage{graphicx}
\usepackage{graphicx}
\usepackage{subcaption}
\usepackage{listings}
\renewcommand{\lstlistingname}{Code}
\usepackage{color}

\definecolor{dkgreen}{rgb}{0,0.6,0}
\definecolor{gray}{rgb}{0.5,0.5,0.5}
\definecolor{mauve}{rgb}{0.58,0,0.82}

\lstset{frame=tb,
  language=Ada,
  aboveskip=3mm,
  belowskip=3mm,
  showstringspaces=false,
  columns=flexible,
  basicstyle={\small\ttfamily},
  numbers=left,
  numberstyle=\tiny\color{gray},
  keywordstyle=\color{blue},
  commentstyle=\color{dkgreen},
  stringstyle=\color{mauve},
  breaklines=true,
  breakatwhitespace=true,
  tabsize=3
}

\usepackage{sectsty} % Allows customizing section commands
\allsectionsfont{\normalfont\scshape} % Make all sections centered, the default font and small caps

\usepackage{fancyhdr} % Custom headers and footers
\pagestyle{fancyplain} % Makes all pages in the document conform to the custom headers and footers
\fancyhead{} % No page header - if you want one, create it in the same way as the footers below
\fancyfoot[L]{} % Empty left footer
\fancyfoot[C]{} % Empty center footer
\fancyfoot[R]{\thepage} % Page numbering for right footer
\renewcommand{\headrulewidth}{0pt} % Remove header underlines
\renewcommand{\footrulewidth}{0pt} % Remove footer underlines
\setlength{\headheight}{13.6pt} % Customize the height of the header

\numberwithin{equation}{section} % Number equations within sections (i.e. 1.1, 1.2, 2.1, 2.2 instead of 1, 2, 3, 4)
\numberwithin{figure}{section} % Number figures within sections (i.e. 1.1, 1.2, 2.1, 2.2 instead of 1, 2, 3, 4)
\numberwithin{table}{section} % Number tables within sections (i.e. 1.1, 1.2, 2.1, 2.2 instead of 1, 2, 3, 4)

\setlength\parindent{0pt} % Removes all indentation from paragraphs - comment this line for an assignment with lots of text

%----------------------------------------------------------------------------------------
%	TITLE SECTION
%----------------------------------------------------------------------------------------

\newcommand{\horrule}[1]{\rule{\linewidth}{#1}} % Create horizontal rule command with 1 argument of height

\title{	
\normalfont \normalsize 
\textsc{ADA Programming Language Training} \\ [25pt] % Your university, school and/or department name(s)
\horrule{0.5pt} \\[0.4cm] % Thin top horizontal rule
\huge Section 02 \\ Packages 01\\ % The assignment title
\horrule{2pt} \\[0.5cm] % Thick bottom horizontal rule
}

\author{Pedro Fernando Flores Palmeros} % Your name

\date{} % Today's date or a custom date

\begin{document}

\maketitle % Print the title         
 
%----------------------------------------------------------------------------------------
%	PROBLEM 1
%----------------------------------------------------------------------------------------

\section{Introduction}
In the prevvious example the Week package only had declarations and no body. In a package specification, you cannot declare bodies. Those have to be in the package body.\\

The next code is \verb|operations.ads| this code is just the manifold or the skeleton of the package, only the function signature is declared but not implemented. In this example there are two main functions the first one is \verb|Increment_By| that has two parameter, the fist one is \verb|I| and it is needed, and the second parameter is \verb|Incr| this parameter is optional, if it is not provided by the user, then it will use its default value as 0. The second function is \verb|Get_Increment_Value| this function does not have parameters and returns an Integer.\\

\begin{lstlisting}[caption = {operations.ads}, label = {operations_ads}]
package Operations is 

	-- Declaration 
	function Increment_By
		(I    : Integer;
		 Incr : Integer := 0) return Integer;

	function Get_Increment_Value return Integer;

end Operations;
\end{lstlisting}

The next code is \verb|operations.adb| in this code is the implementation of the functions declared in code~\ref{operations_ads}. Coincidentally, introducing a body allows us to put the \verb|Last_Increment| variable in the body and make then inaccessible to the user of the \verb|Operations| package, providing the fist form of encapuslation. This works because entities declared in the body are \textit{only} visible in the body. \\

\begin{lstlisting}[caption = {operations.adb}]
package body Operations is

	Last_Increment : Integer := 1;

	function Increment_By
		( I    : Integer;
		  Incr : Integer := 0) return Integer is
	begin
		if Incr /= 0 then
			Last_Increment := Incr;
		end if; 

		return I + Last_Increment;
	end Increment_By;

	function Get_Increment_Value return Integer is
	begin 
		return Last_Increment;
	end Get_Increment_Value;
	
end Operations;
\end{lstlisting}

In the \verb|main| the \verb|Operations| package is used and a procedure is declarad and implemented only to display de values. Note that in line 22 the procedure \verb|Increment_By| is used with two arguments and in line 17 is invoked only with one value. 

\begin{lstlisting}[caption = {operations.adb}]
with Ada.Text_IO; use Ada.Text_IO;
with Operations;

procedure Main is
	use Operations;
	I : Integer := 0;
	R : Integer;

	procedure Display_Udate_Values is
		Incr : constant Integer := Get_Increment_Value;
	begin  
		Put_Line(Integer'Image(I) & " incremented by "
			 & Integer'Image(Incr) & " is " & Integer'Image(R));
		I := R;
	end Display_Udate_Values;
begin 
	R := Increment_By(I);
	Display_Udate_Values;
	R := Increment_By(I);
	Display_Udate_Values;

	R := Increment_By(I,5);
	Display_Udate_Values;
	R := Increment_By(I);
	Display_Udate_Values;

	R := Increment_By(I, 10);
	Display_Udate_Values;
	R := Increment_By(I);
	Display_Udate_Values;
end Main;
\end{lstlisting}


\end{document}