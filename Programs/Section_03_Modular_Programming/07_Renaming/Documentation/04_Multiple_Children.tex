%%%%%%%%%%%%%%%%%%%%%%%%%%%%%%%%%%%%%%%%%
% Short Sectioned Assignment
% LaTeX Template
% Version 1.0 (5/5/12)
%
% This template has been downloaded from:
% http://www.LaTeXTemplates.com
%
% Original author:
% Frits Wenneker (http://www.howtotex.com)
%
% License:
% CC BY-NC-SA 3.0 (http://creativecommons.org/licenses/by-nc-sa/3.0/)
%
%%%%%%%%%%%%%%%%%%%%%%%%%%%%%%%%%%%%%%%%%

%----------------------------------------------------------------------------------------
%	PACKAGES AND OTHER DOCUMENT CONFIGURATIONS
%----------------------------------------------------------------------------------------

\documentclass[paper=a4, fontsize=11pt]{scrartcl} % A4 paper and 11pt font size

\usepackage[T1]{fontenc} % Use 8-bit encoding that has 256 glyphs
\usepackage{fourier} % Use the Adobe Utopia font for the document - comment this line to return to the LaTeX default
%\usepackage[spanish]{babel} % English language/hyphenation
\usepackage[utf8]{inputenc}
\usepackage{amsmath,amsfonts,amsthm} % Math packages
\usepackage{graphicx}
\usepackage{graphicx}
\usepackage{subcaption}
\usepackage{listings}
\renewcommand{\lstlistingname}{Code}
\usepackage{color}

\definecolor{dkgreen}{rgb}{0,0.6,0}
\definecolor{gray}{rgb}{0.5,0.5,0.5}
\definecolor{mauve}{rgb}{0.58,0,0.82}

\lstset{frame=tb,
  language=Ada,
  aboveskip=3mm,
  belowskip=3mm,
  showstringspaces=false,
  columns=flexible,
  basicstyle={\small\ttfamily},
  numbers=left,
  numberstyle=\tiny\color{gray},
  keywordstyle=\color{blue},
  commentstyle=\color{dkgreen},
  stringstyle=\color{mauve},
  breaklines=true,
  breakatwhitespace=true,
  tabsize=3
}

\usepackage{sectsty} % Allows customizing section commands
\allsectionsfont{\normalfont\scshape} % Make all sections centered, the default font and small caps

\usepackage{fancyhdr} % Custom headers and footers
\pagestyle{fancyplain} % Makes all pages in the document conform to the custom headers and footers
\fancyhead{} % No page header - if you want one, create it in the same way as the footers below
\fancyfoot[L]{} % Empty left footer
\fancyfoot[C]{} % Empty center footer
\fancyfoot[R]{\thepage} % Page numbering for right footer
\renewcommand{\headrulewidth}{0pt} % Remove header underlines
\renewcommand{\footrulewidth}{0pt} % Remove footer underlines
\setlength{\headheight}{13.6pt} % Customize the height of the header

\numberwithin{equation}{section} % Number equations within sections (i.e. 1.1, 1.2, 2.1, 2.2 instead of 1, 2, 3, 4)
\numberwithin{figure}{section} % Number figures within sections (i.e. 1.1, 1.2, 2.1, 2.2 instead of 1, 2, 3, 4)
\numberwithin{table}{section} % Number tables within sections (i.e. 1.1, 1.2, 2.1, 2.2 instead of 1, 2, 3, 4)

\setlength\parindent{0pt} % Removes all indentation from paragraphs - comment this line for an assignment with lots of text

%----------------------------------------------------------------------------------------
%	TITLE SECTION
%----------------------------------------------------------------------------------------

\newcommand{\horrule}[1]{\rule{\linewidth}{#1}} % Create horizontal rule command with 1 argument of height

\title{	
\normalfont \normalsize 
\textsc{ADA Programming Language Training} \\ [25pt] % Your university, school and/or department name(s)
\horrule{0.5pt} \\[0.4cm] % Thin top horizontal rule
\huge Section 02 \\ Child packages\\ % The assignment title
\horrule{2pt} \\[0.5cm] % Thick bottom horizontal rule
}

\author{Pedro Fernando Flores Palmeros} % Your name

\date{} % Today's date or a custom date

\begin{document}

\maketitle % Print the title         
 
%----------------------------------------------------------------------------------------
%	PROBLEM 1
%----------------------------------------------------------------------------------------

\section{Introduction}
Packages can be used to create hierarchies. This is achieved by using child packages, which extend the functionality of their parent package. An example of this is the \verb|Ada.Text_IO| package, the parent package is called \verb|Ada| and the child package is \verb|Text_IO|. In the next codes is illustrated the creation and implementation of the \textit{child packages}.\\

For this example package is \verb|week| is created into a \verb|.ads| file this is the main package and it will contain only some declarations.



\begin{lstlisting}[caption = {week.ads}, label = {program_01}]
package Week is

	Mon : constant String := "Monday";
	Tue : constant String := "Tuesday";
	Wed : constant String := "Wednesday"; 
	Thr : constant String := "Thursday";
	Fri : constant String := "Friday"; 
	Sat : constant String := "Saturday"; 
	Sun : constant String := "Sunday"; 

end Week;
\end{lstlisting}

The parent child is the \verb|Week| package and the child package is going to be \verb|Week.Child|, the child package is divided into declaration (\verb|.ads| file) and implementation (\verb|.adb| file) both of them are shown in the next snippets.

\begin{lstlisting}[caption = {week-child.ads}, label = {program_02}]
package Week.Child is

	function Get_First_Of_Week return String; 
	
end Week.Child;
\end{lstlisting}

\begin{lstlisting}[caption = {week-child.adb}, label = {program_03}]
package body Week.Child is 

	function Get_First_Of_Week return String is
	begin 

		return Mon;

	end Get_First_Of_Week;

end Week.Child;
\end{lstlisting}

Observe that in the child package the data and variables from the parent package are available. For example in the line 6 of the child package the function \verb|Get_First_Of_Week| returns the \verb|Mon| variable that was declared in the parent package.\\

The main procedure might has to add both of them the parent and the child package

\begin{lstlisting}[caption = {main.adb}, label = {program_04}]
with Ada.Text_IO; use Ada.Text_IO; 
with Week.Child; use Week.Child;

procedure Main is
begin
	Put_Line("First day of the week is: " & Get_First_Of_Week);
end Main;
\end{lstlisting}

In \verb|main.adb| the line 6 can be changed by\\
\verb|Put_Line("First day of the week is: " & Week.Child.Get_First_Of_Week);|

\end{document}