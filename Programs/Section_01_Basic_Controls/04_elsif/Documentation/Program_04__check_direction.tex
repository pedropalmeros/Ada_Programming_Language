%%%%%%%%%%%%%%%%%%%%%%%%%%%%%%%%%%%%%%%%%
% Short Sectioned Assignment
% LaTeX Template
% Version 1.0 (5/5/12)
%
% This template has been downloaded from:
% http://www.LaTeXTemplates.com
%
% Original author:
% Frits Wenneker (http://www.howtotex.com)
%
% License:
% CC BY-NC-SA 3.0 (http://creativecommons.org/licenses/by-nc-sa/3.0/)
%
%%%%%%%%%%%%%%%%%%%%%%%%%%%%%%%%%%%%%%%%%

%----------------------------------------------------------------------------------------
%	PACKAGES AND OTHER DOCUMENT CONFIGURATIONS
%----------------------------------------------------------------------------------------

\documentclass[paper=a4, fontsize=11pt]{scrartcl} % A4 paper and 11pt font size

\usepackage[T1]{fontenc} % Use 8-bit encoding that has 256 glyphs
\usepackage{fourier} % Use the Adobe Utopia font for the document - comment this line to return to the LaTeX default
%\usepackage[spanish]{babel} % English language/hyphenation
\usepackage[utf8]{inputenc}
\usepackage{amsmath,amsfonts,amsthm} % Math packages
\usepackage{graphicx}
\usepackage{graphicx}
\usepackage{subcaption}
\usepackage{listings}
\usepackage{color}

\definecolor{dkgreen}{rgb}{0,0.6,0}
\definecolor{gray}{rgb}{0.5,0.5,0.5}
\definecolor{mauve}{rgb}{0.58,0,0.82}

\lstset{frame=tb,
  language=Ada,
  aboveskip=3mm,
  belowskip=3mm,
  showstringspaces=false,
  columns=flexible,
  basicstyle={\small\ttfamily},
  numbers=left,
  numberstyle=\tiny\color{gray},
  keywordstyle=\color{blue},
  commentstyle=\color{dkgreen},
  stringstyle=\color{mauve},
  breaklines=true,
  breakatwhitespace=true,
  tabsize=3
}

\usepackage{sectsty} % Allows customizing section commands
\allsectionsfont{\normalfont\scshape} % Make all sections centered, the default font and small caps

\usepackage{fancyhdr} % Custom headers and footers
\pagestyle{fancyplain} % Makes all pages in the document conform to the custom headers and footers
\fancyhead{} % No page header - if you want one, create it in the same way as the footers below
\fancyfoot[L]{} % Empty left footer
\fancyfoot[C]{} % Empty center footer
\fancyfoot[R]{\thepage} % Page numbering for right footer
\renewcommand{\headrulewidth}{0pt} % Remove header underlines
\renewcommand{\footrulewidth}{0pt} % Remove footer underlines
\setlength{\headheight}{13.6pt} % Customize the height of the header

\numberwithin{equation}{section} % Number equations within sections (i.e. 1.1, 1.2, 2.1, 2.2 instead of 1, 2, 3, 4)
\numberwithin{figure}{section} % Number figures within sections (i.e. 1.1, 1.2, 2.1, 2.2 instead of 1, 2, 3, 4)
\numberwithin{table}{section} % Number tables within sections (i.e. 1.1, 1.2, 2.1, 2.2 instead of 1, 2, 3, 4)

\setlength\parindent{0pt} % Removes all indentation from paragraphs - comment this line for an assignment with lots of text

%----------------------------------------------------------------------------------------
%	TITLE SECTION
%----------------------------------------------------------------------------------------

\newcommand{\horrule}[1]{\rule{\linewidth}{#1}} % Create horizontal rule command with 1 argument of height

\title{	
\normalfont \normalsize 
\textsc{ADA Programming Language Training} \\ [25pt] % Your university, school and/or department name(s)
\horrule{0.5pt} \\[0.4cm] % Thin top horizontal rule
\huge Program 04\\ Check Direction \\ % The assignment title
\horrule{2pt} \\[0.5cm] % Thick bottom horizontal rule
}

\author{Pedro Fernando Flores Palmeros} % Your name

\date{} % Today's date or a custom date

\begin{document}

\maketitle % Print the title         
 
%----------------------------------------------------------------------------------------
%	PROBLEM 1
%----------------------------------------------------------------------------------------

\section{Introduction}
The main syntaxis of the \verb|if| control structure is the next

\begin{lstlisting}
if <condition> then
	statement 01;
	...
	statement n;
elsif <condition 1> then
	statement 01;
	...
	statement m;
elsif <condition q> then
	statement 01;
	...
	statement p;
else
	statement 01;
	...
	statement k;
end if;
\end{lstlisting}
in this case \verb|<condition>| is the condition to be tested it does not require the \verb|<>| symbols.

\section{Code}
\begin{lstlisting}
with Ada.Text_IO; use Ada.Text_IO;
with Ada.Integer_Text_IO; use Ada.Integer_Text_IO;

procedure Check_Direction is 
    N : Integer;
begin
    Put("enter an integer value");
    Get(N);
    Put(N);

    if N=0 or N=360 then
        Put_Line("  is due north");
    elsif N in 1..89 then
        Put_Line("  is in the northeast quadrant");
    elsif N in 90 then
        Put_Line(" is due east");
    elsif N in 91 ..179 then
        Put_Line(" is in the southeast quadrant" );
    elsif N in 180 then
        Put_Line(" is due south" );
    elsif N in 181..269 then
        Put_Line(" is in the southwest quadrant" );
    elsif N = 270 then
        Put_Line(" is due west");
    elsif N in 271 .. 359 then
        Put_Line(" is in the northwest quadrant" );
    else
        Put_Line(" is not in the range 0 ... 360" );
    end if;
end Check_Direction;


\end{lstlisting}

\section{Main parts}
Observe that in this case in line 1 and 2 the packages that are going to be used are declared, and that is why in the lines 8, 15, 16, 17 the commands are without \verb|Ada.Text_IO|. For further information in the first program is a detailed explanation of it. 

\subsection{Get(N)}
This function is part of the package \verb|Ada.Text_IO|, and it is useful for reading Integers from the keyboard, observe that a variable of Integer type has to be sent in the function argument. 

\subsection{Variable declaration}
The variables and subsfunctions that are going to be used and implemented in the main procedure has to be defined in the lines between the procedure declaration and the begin of the implementation of the procedure. In this particular case between line 4 and 5.

\subsection{if - elsif - else  directives}
Sometimes due the nature of the algorithm or problem to be solved it is required to use multiple conditions to evaluate a certain event, in this case the directive to be used is the \verb|elsif|. \\

With this directive many conditions can be nested. In which if the first condition is not filled the second condition is tested and if the second is not filled, then it goes through the rest fo the conditions, if no condition has been accomplished then the conditions attached to the else are executed. 




\end{document}