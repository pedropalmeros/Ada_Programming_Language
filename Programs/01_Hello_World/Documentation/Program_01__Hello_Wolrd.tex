%%%%%%%%%%%%%%%%%%%%%%%%%%%%%%%%%%%%%%%%%
% Short Sectioned Assignment
% LaTeX Template
% Version 1.0 (5/5/12)
%
% This template has been downloaded from:
% http://www.LaTeXTemplates.com
%
% Original author:
% Frits Wenneker (http://www.howtotex.com)
%
% License:
% CC BY-NC-SA 3.0 (http://creativecommons.org/licenses/by-nc-sa/3.0/)
%
%%%%%%%%%%%%%%%%%%%%%%%%%%%%%%%%%%%%%%%%%

%----------------------------------------------------------------------------------------
%	PACKAGES AND OTHER DOCUMENT CONFIGURATIONS
%----------------------------------------------------------------------------------------

\documentclass[paper=a4, fontsize=11pt]{scrartcl} % A4 paper and 11pt font size

\usepackage[T1]{fontenc} % Use 8-bit encoding that has 256 glyphs
\usepackage{fourier} % Use the Adobe Utopia font for the document - comment this line to return to the LaTeX default
%\usepackage[spanish]{babel} % English language/hyphenation
\usepackage[utf8]{inputenc}
\usepackage{amsmath,amsfonts,amsthm} % Math packages
\usepackage{graphicx}
\usepackage{graphicx}
\usepackage{subcaption}
\usepackage{listings}
\usepackage{color}

\definecolor{dkgreen}{rgb}{0,0.6,0}
\definecolor{gray}{rgb}{0.5,0.5,0.5}
\definecolor{mauve}{rgb}{0.58,0,0.82}

\lstset{frame=tb,
  language=Ada,
  aboveskip=3mm,
  belowskip=3mm,
  showstringspaces=false,
  columns=flexible,
  basicstyle={\small\ttfamily},
  numbers=left,
  numberstyle=\tiny\color{gray},
  keywordstyle=\color{blue},
  commentstyle=\color{dkgreen},
  stringstyle=\color{mauve},
  breaklines=true,
  breakatwhitespace=true,
  tabsize=3
}

\usepackage{sectsty} % Allows customizing section commands
\allsectionsfont{\normalfont\scshape} % Make all sections centered, the default font and small caps

\usepackage{fancyhdr} % Custom headers and footers
\pagestyle{fancyplain} % Makes all pages in the document conform to the custom headers and footers
\fancyhead{} % No page header - if you want one, create it in the same way as the footers below
\fancyfoot[L]{} % Empty left footer
\fancyfoot[C]{} % Empty center footer
\fancyfoot[R]{\thepage} % Page numbering for right footer
\renewcommand{\headrulewidth}{0pt} % Remove header underlines
\renewcommand{\footrulewidth}{0pt} % Remove footer underlines
\setlength{\headheight}{13.6pt} % Customize the height of the header

\numberwithin{equation}{section} % Number equations within sections (i.e. 1.1, 1.2, 2.1, 2.2 instead of 1, 2, 3, 4)
\numberwithin{figure}{section} % Number figures within sections (i.e. 1.1, 1.2, 2.1, 2.2 instead of 1, 2, 3, 4)
\numberwithin{table}{section} % Number tables within sections (i.e. 1.1, 1.2, 2.1, 2.2 instead of 1, 2, 3, 4)

\setlength\parindent{0pt} % Removes all indentation from paragraphs - comment this line for an assignment with lots of text

%----------------------------------------------------------------------------------------
%	TITLE SECTION
%----------------------------------------------------------------------------------------

\newcommand{\horrule}[1]{\rule{\linewidth}{#1}} % Create horizontal rule command with 1 argument of height

\title{	
\normalfont \normalsize 
\textsc{ADA Programming Language Training} \\ [25pt] % Your university, school and/or department name(s)
\horrule{0.5pt} \\[0.4cm] % Thin top horizontal rule
\huge Program 01 \\ Hello World \\ % The assignment title
\horrule{2pt} \\[0.5cm] % Thick bottom horizontal rule
}

\author{Pedro Fernando Flores Palmeros} % Your name

\date{} % Today's date or a custom date

\begin{document}

\maketitle % Print the title         
 
%----------------------------------------------------------------------------------------
%	PROBLEM 1
%----------------------------------------------------------------------------------------

\section{Introduction}
The main pupose of this document is just to explain main components of the code. It is not go deeper, if further information is required books can be used.

\section{Code}



\begin{lstlisting}
-- hello_world.adb

with Ada.Text_IO;

procedure hello_world is

begin
    Ada.Text_IO.Put_Line("Hello World!");
    Ada.Text_IO.Put_Line("It's a wonderful day!");
    Ada.Text_IO.New_Line;
end hello_world;
\end{lstlisting}

\section{Important details and commands}

\subsection{Comments}
In ADA the comments marked with \verb+--+ and there are only comments of  a single line. 

\subsection{procedure hello world}
A procedure can be seen like a function or special kind of it, in this case it is the main function and it is named \verb+hello_wolrd+.\\

If some variables might be or \textit{subfunctions} are needed they have to be plaecd in the space between the declarartion of the function (which in this case is given by \verb+procedure hello_world is+ and the \verb+begin+ command that in this particular case is at line 7.\\

The end of the procedure is in line 11 observe that it has to begin with the word \verb+end+ and then the name of the procedure to be finished, hence in this parcular case is \verb+end hello_world+. 

\subsection{Printing in the console}

\verb|Ada.Text_IO.Put_Line("|\textit{text}\verb|")| with this command the text in italic is printed in the command line and after printing the contenet between the double quotes a new line will be commanded, hence the cursor is set one line down and in the left of the command line.\\

\verb|Ada.Text_IO.Put("|\textit{text}\verb|")| with this command the text in italic is printed in the command line but the main difference between this command and the one before is that this one \textbf{DOES NOT ADD A NEW LINE}.\\

\verb|Ada.Text_IO.New_Line| with this command no text is printed on the screen, only new line is commanded.

\section{Hello World option 2}
 
\begin{lstlisting}
-- hello_world.adb

with Ada.Text_IO; use Ada.Text_IO;

procedure hello_world is

begin
    Put_Line("Hello World!");
    Put_Line("It's a wonderful day!");
    New_Line;
end hello_world;
\end{lstlisting}

The main difference between this program and the first one is the that in the line 3 the part \verb|use Ada.Text_IO;| helps in the subsequent lines and observe that now to invoque the fucntions \verb|Put_Line|, \verb|Put| and \verb|New_Line| it is not needed the first part of the command \verb|Ada.Text_IO| since it has been declared previously.\\

\end{document}