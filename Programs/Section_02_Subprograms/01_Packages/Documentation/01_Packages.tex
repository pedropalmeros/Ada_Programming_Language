%%%%%%%%%%%%%%%%%%%%%%%%%%%%%%%%%%%%%%%%%
% Short Sectioned Assignment
% LaTeX Template
% Version 1.0 (5/5/12)
%
% This template has been downloaded from:
% http://www.LaTeXTemplates.com
%
% Original author:
% Frits Wenneker (http://www.howtotex.com)
%
% License:
% CC BY-NC-SA 3.0 (http://creativecommons.org/licenses/by-nc-sa/3.0/)
%
%%%%%%%%%%%%%%%%%%%%%%%%%%%%%%%%%%%%%%%%%

%----------------------------------------------------------------------------------------
%	PACKAGES AND OTHER DOCUMENT CONFIGURATIONS
%----------------------------------------------------------------------------------------

\documentclass[paper=a4, fontsize=11pt]{scrartcl} % A4 paper and 11pt font size

\usepackage[T1]{fontenc} % Use 8-bit encoding that has 256 glyphs
\usepackage{fourier} % Use the Adobe Utopia font for the document - comment this line to return to the LaTeX default
%\usepackage[spanish]{babel} % English language/hyphenation
\usepackage[utf8]{inputenc}
\usepackage{amsmath,amsfonts,amsthm} % Math packages
\usepackage{graphicx}
\usepackage{graphicx}
\usepackage{subcaption}
\usepackage{listings}
\renewcommand{\lstlistingname}{Code}
\usepackage{color}

\definecolor{dkgreen}{rgb}{0,0.6,0}
\definecolor{gray}{rgb}{0.5,0.5,0.5}
\definecolor{mauve}{rgb}{0.58,0,0.82}

\lstset{frame=tb,
  language=Ada,
  aboveskip=3mm,
  belowskip=3mm,
  showstringspaces=false,
  columns=flexible,
  basicstyle={\small\ttfamily},
  numbers=left,
  numberstyle=\tiny\color{gray},
  keywordstyle=\color{blue},
  commentstyle=\color{dkgreen},
  stringstyle=\color{mauve},
  breaklines=true,
  breakatwhitespace=true,
  tabsize=3
}

\usepackage{sectsty} % Allows customizing section commands
\allsectionsfont{\normalfont\scshape} % Make all sections centered, the default font and small caps

\usepackage{fancyhdr} % Custom headers and footers
\pagestyle{fancyplain} % Makes all pages in the document conform to the custom headers and footers
\fancyhead{} % No page header - if you want one, create it in the same way as the footers below
\fancyfoot[L]{} % Empty left footer
\fancyfoot[C]{} % Empty center footer
\fancyfoot[R]{\thepage} % Page numbering for right footer
\renewcommand{\headrulewidth}{0pt} % Remove header underlines
\renewcommand{\footrulewidth}{0pt} % Remove footer underlines
\setlength{\headheight}{13.6pt} % Customize the height of the header

\numberwithin{equation}{section} % Number equations within sections (i.e. 1.1, 1.2, 2.1, 2.2 instead of 1, 2, 3, 4)
\numberwithin{figure}{section} % Number figures within sections (i.e. 1.1, 1.2, 2.1, 2.2 instead of 1, 2, 3, 4)
\numberwithin{table}{section} % Number tables within sections (i.e. 1.1, 1.2, 2.1, 2.2 instead of 1, 2, 3, 4)

\setlength\parindent{0pt} % Removes all indentation from paragraphs - comment this line for an assignment with lots of text

%----------------------------------------------------------------------------------------
%	TITLE SECTION
%----------------------------------------------------------------------------------------

\newcommand{\horrule}[1]{\rule{\linewidth}{#1}} % Create horizontal rule command with 1 argument of height

\title{	
\normalfont \normalsize 
\textsc{ADA Programming Language Training} \\ [25pt] % Your university, school and/or department name(s)
\horrule{0.5pt} \\[0.4cm] % Thin top horizontal rule
\huge Section 02 \\ Packages 01\\ % The assignment title
\horrule{2pt} \\[0.5cm] % Thick bottom horizontal rule
}

\author{Pedro Fernando Flores Palmeros} % Your name

\date{} % Today's date or a custom date

\begin{document}

\maketitle % Print the title         
 
%----------------------------------------------------------------------------------------
%	PROBLEM 1
%----------------------------------------------------------------------------------------

\section{Introduction}
In the previous section simple standalone subprograms were presented. It is easy to see that the programming style from the first section is not useful for scaling up for real-world applications. A better way to structure the code into modular and distinct units is needed.\\

Ada encourages the separation of programs into multiple packages and sub-packages, providing many tools to a  programmer on a quest for a perfectly organized code-base. 

Here is an example of a package in ADA

\begin{lstlisting}[caption = {week.ads}]
package Week is
	Mon : constant String := "Monday";
	Tue : constant String := "Tuesday";
	Wed : constant String := "Wednesday";
	Thu : constant String := "Thursday";
	Fri : constant String := "Friday";
	Sat : constant String := "Saturday";
	Sun : constant String := "Sunday";
end Week;
\end{lstlisting}

The \verb|Code 1| is package, the first aspect to have on mind is that the file extension has changed to \verb|.ads|  which is a \verb|spect| ADA file. This do not do anything by itself, it has to be invoked from a \verb|main| or other program. 

\begin{lstlisting}[caption = {main.adb}]
with Ada.Text_IO; use Ada.Text_IO;
with Week;
-- References the Week package,
-- and ads a dependency from Main to Week

procedure Main is 
begin  
	Put_Line("First day of the week is " & Week.Mon);
end Main;
\end{lstlisting}

The \verb|Code 2| snippet is the main code, in this code at line 8 is being invoked the week package and specifically the Monday day. \\

Packages let make your code modular, separating you programs into semantically significant unit. Additionally the separation of a package's specification from its body can reduce compilation time.\\

While the \verb|with| clause indicates a dependency, you can see in the example that you still need to prefix the referencing of entities from the Week package by the name of the package. If we had included a \verb|use| Week clause, then such a prefix would not have been necessary.\\

Accessing entities from a package uses the dot notation, A.B which is the same notation as the one used to access record files. \\

A \verb|with| clause can \textit{only} appear in the prelude of a compilation unit (before the reserved word such as \verb|procedure|, that makrs the beginning of the unit). It is not allowed anywhere else. This rule is only needed for methodological reasons: the person reading your code should be able to see immediately which units the code depends on. 

\begin{lstlisting}[caption = {main.adb}]
with Ada.Text_IO; use Ada.Text_IO;
with Week; use Week;
-- References the Week package,
-- and ads a dependency from Main to Week

procedure Main is 
begin  
	Put_Line("First day of the week is " & Mon);
end Main;
\end{lstlisting}



\end{document}