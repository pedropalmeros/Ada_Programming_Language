%%%%%%%%%%%%%%%%%%%%%%%%%%%%%%%%%%%%%%%%%
% Short Sectioned Assignment
% LaTeX Template
% Version 1.0 (5/5/12)
%
% This template has been downloaded from:
% http://www.LaTeXTemplates.com
%
% Original author:
% Frits Wenneker (http://www.howtotex.com)
%
% License:
% CC BY-NC-SA 3.0 (http://creativecommons.org/licenses/by-nc-sa/3.0/)
%
%%%%%%%%%%%%%%%%%%%%%%%%%%%%%%%%%%%%%%%%%

%----------------------------------------------------------------------------------------
%	PACKAGES AND OTHER DOCUMENT CONFIGURATIONS
%----------------------------------------------------------------------------------------

\documentclass[paper=a4, fontsize=11pt]{scrartcl} % A4 paper and 11pt font size

\usepackage[T1]{fontenc} % Use 8-bit encoding that has 256 glyphs
\usepackage{fourier} % Use the Adobe Utopia font for the document - comment this line to return to the LaTeX default
%\usepackage[spanish]{babel} % English language/hyphenation
\usepackage[utf8]{inputenc}
\usepackage{amsmath,amsfonts,amsthm} % Math packages
\usepackage{graphicx}
\usepackage{graphicx}
\usepackage{subcaption}
\usepackage{listings}
\renewcommand{\lstlistingname}{Code}
\usepackage{color}



\definecolor{dkgreen}{rgb}{0,0.6,0}
\definecolor{gray}{rgb}{0.5,0.5,0.5}
\definecolor{mauve}{rgb}{0.58,0,0.82}

\lstset{frame=tb,
  language=Ada,
  aboveskip=3mm,
  belowskip=3mm,
  showstringspaces=false,
  columns=flexible,
  basicstyle={\small\ttfamily},
  numbers=left,
  numberstyle=\tiny\color{gray},
  keywordstyle=\color{blue},
  commentstyle=\color{dkgreen},
  stringstyle=\color{mauve},
  breaklines=true,
  breakatwhitespace=true,
  tabsize=3
}

\usepackage{sectsty} % Allows customizing section commands
\allsectionsfont{\normalfont\scshape} % Make all sections centered, the default font and small caps

\usepackage{fancyhdr} % Custom headers and footers
\pagestyle{fancyplain} % Makes all pages in the document conform to the custom headers and footers
\fancyhead{} % No page header - if you want one, create it in the same way as the footers below
\fancyfoot[L]{} % Empty left footer
\fancyfoot[C]{} % Empty center footer
\fancyfoot[R]{\thepage} % Page numbering for right footer
\renewcommand{\headrulewidth}{0pt} % Remove header underlines
\renewcommand{\footrulewidth}{0pt} % Remove footer underlines
\setlength{\headheight}{13.6pt} % Customize the height of the header

\numberwithin{equation}{section} % Number equations within sections (i.e. 1.1, 1.2, 2.1, 2.2 instead of 1, 2, 3, 4)
\numberwithin{figure}{section} % Number figures within sections (i.e. 1.1, 1.2, 2.1, 2.2 instead of 1, 2, 3, 4)
\numberwithin{table}{section} % Number tables within sections (i.e. 1.1, 1.2, 2.1, 2.2 instead of 1, 2, 3, 4)

\setlength\parindent{0pt} % Removes all indentation from paragraphs - comment this line for an assignment with lots of text

\newcommand{\refCode}[1]{Code.(\ref{#1})}


%----------------------------------------------------------------------------------------
%	TITLE SECTION
%----------------------------------------------------------------------------------------

\newcommand{\horrule}[1]{\rule{\linewidth}{#1}} % Create horizontal rule command with 1 argument of height



\title{	
\normalfont \normalsize 
\textsc{ADA Programming Language Training} \\ [25pt] % Your university, school and/or department name(s)
\horrule{0.5pt} \\[0.4cm] % Thin top horizontal rule
\huge Section 02 - Subprograms \\ Subprograms 01\\ % The assignment title
\horrule{2pt} \\[0.5cm] % Thick bottom horizontal rule
}

\author{Pedro Fernando Flores Palmeros} % Your name

\date{} % Today's date or a custom date

\begin{document}

\maketitle % Print the title         
 
%----------------------------------------------------------------------------------------
%	PROBLEM 1
%----------------------------------------------------------------------------------------

\section{Introduction}

Subprograms can be renamed by using the \textbf{renames} keyword and declaring a new name for a subprogram:

\begin{lstlisting}[caption = {renames syntax}, label = {renames}]
procedure New_Proc renames Original_Proc;
\end{lstlisting}


ADA is known as safety-focused language. There are many ways this is realized but two important points are: 

\begin{itemize}
	\item Ada makes the user specify as much as possible about the behavior expected for the program, so that the compiler can warn or reject if there is an inconsistency.
	\item Ada provides a variety of techniques for achieve the two designs goals above. A subprogram parameter can be specified a mode, which is one of the following:
	\begin{itemize}
		\item \textbf{in} Parameter can only be read, not wri´tten.
		\item \textbf{out} Parameter can be written to, then read.
		\item \textbf{in out} Parameter can be both read and written.
	\end{itemize}
\end{itemize}
The default mode for parameters is \textbf{in}; so far, most of the examples have been suing \textbf{in} parameters.\\

In the \refCode{main_adb} is executed the case in which a procedure named \verb|In_Out_Params| has two arguments, both of them are of type \textbf{in out}. Observer that \textbf{it is a procedure, hence it should not use return, but it is modifying the variables and setting them as outputs}. It might be like using references like in C/C++.

\begin{lstlisting}[caption = {main.adb}, label = {main_adb}]
with Ada.Text_IO; use Ada.Text_IO;

procedure In_Out_Paramters is
	procedure Swap(A, B: in out Integer) is 
		Tmp : Integer;
	begin 
		Tmp := A; 
		A   := B;
		B   := Tmp;
	end Swap;
	A : Integer := 12;
	B : Integer := 44;
begin 
	Swap(A,B);

	-- prints 44
	Put_Line(Integer'Image(A));
	end In_Out_Paramters;
\end{lstlisting}

An \textbf{in out} parameter will allow read and write access to the object passed as parameter, so in the example above, we can see that A is modified after the call to Swap.


\end{document}